\chapter{Future Work}
During the work of this thesis, a few issues arose that could be interesting to look into for future work. This chapter could be seen as a list of suggestions to future projects regarding the Husky A200 UGV at UiA, to better equip this for such a warehouse automation task or other relevant tasks a future student might explore. More work could be put into making the entire system more robust and flexible. For example, the vision system would need adjustments to achieve a more acceptable pick and place accuracy. Definition of new tags and grasping poses for object picking is not gathered in one parameter file, increasing the complexity of adding object definitions to the pick and place system. Nevertheless, it might be better to completely swap out some of these systems to achieve greater performance.

As explained several times in chapter \ref{sec:Discussion}, the navigational performance of the setup is severely impacted by the skidding locomotion of this mobile robotic platform. The suggestion is therefore to not use the Husky A200 as a platform for a warehouse automation system, as a warehouse usually consists of flat hard concrete floors and would not require the extra traction provided by this type of locomotion mechanism. With that being said, this robotic system is intended for educational purposes and this type of locomotion provides a challenge in navigational performance that could be explored in the future. Normally, the system would navigate successfully, and usually, the issue arose when the navigational system would spin to recover it's pose, there might be possibilities to turn off this spinning behaviour.

The limited reach, payload, gripping strength and 5-DOF of the Interbotix VX300 manipulator places higher requirements on the navigation system and gives the robotic system less flexibility in terms of what objects it could pick and how it could pick them. It is therefore suggested to look into the possibilities of mounting a larger manipulator to the mobile robot. For example, ClearPath provides Universal Robot UR5 manipulators together with Husky A200 upon request, which means that it would be possible to mount this kind of manipulator to the robot. For this to be done though, a new power distribution system would have to be made for the Husky. The UR5 manipulator requires a dedicated controller to operate. This controller also contains the DC-power supply for the manipulator. For this manipulator to be mounted on an UGV, this controller must be powered through DC-power. According to the Universal Robots CB3 battery supply installation manual \cite{ur5_battery_manual}, the UR5 could draw up to 50A at "peak conditions". Looking at the Husky's specifications in appendix \ref{tab:husky:a200:specs}, it is clear that the built in 24V power distribution on the Husky, with a 5A circuit breaker, is insufficient.

As mentioned in section \ref{sec:M:PAP:MachineVision}, a fiducial tag-based vision system was chosen in order to ease implementation and to not shift the focus away from the scope of the thesis which is more focused around complete system implementation. However, the pick and place system is made to not know how the detected object is found, it only looks for the coordinate frame it is tasked with picking. This means that any computer vision principle could be used to find the pose of an object to pick, as long as this computer vision system publishes a ROS 2 coordinate frame of the detected object. Therefore, it could be interesting for future projects to focus on implementing some CNN-based object detection and pose estimation method, like the ones presented in section \ref{sec:T:MV:CNN-based} or something similar.

As discussed in section \ref{sec:D:ROS2Distros}, the choice of ROS 2 distributions and the fact that two different ROS 2 distributions is sub-optimal. More work could therefore be put into exploring the possibility of using the latest LTS release Humble Hawksbill. This Distribution is predicted to be supported until May 2027 \cite{ROS2distros}.

% The design of a mobile power system for the CB3 controller and thus, the UR5 manipulator is beyond the scope of this project. The decision was therefore made to swap out the UR5 manipulator with a smaller Interbotix VX300 manipulator from Trossen Robotics, illustrated in figure \ref{fig:M:CD:FC:VX300}. This is a significantly smaller robotic arm that is powered through 12V DC-power supply with a standard 2.5mm DC barrel jack. Some of it's key specifications is shown in table \ref{tab:M:CD:FC:VX300Specs}