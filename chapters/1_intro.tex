\chapter{Introduction}
This chapter provides an overview of the motivation behind this thesis, and the thesis objectives. To start with, the concept of warehouse automation will be introduced along with a brief introduction to the building blocks required to achieve warehouse automation. Then, the objectives regarding this thesis will be presented, what parts of warehouse automation this thesis covers and what hardware and software components that are needed to fulfil the objective.

% Autonomous robotics has been gaining massive traction over the last decade. This increased interest in autonomous robotics has sparked inventions like Robot Operating System(ROS). ROS, and its successor Robot Operating System 2(ROS 2), is an open source operating system made give developers the tools necessary to implement advanced robotic applications without the need for comprehensive knowledge about the inner workings of for example 3D Light Detection and Ranging(LiDAR) sensors or Permanent Magnet Synchronous Motors(PMSM). This opens the door to develop autonomous warehouse automation systems where autonomous Unmanned Ground Vehicles(UGV), could navigate a standard warehouse and fetch items using a mounted robotic manipulator. This thesis presents a proof of concept for using differential drive robots together with computer vision and robotic manipulators to preform autonomous warehouse automation tasks.


% AJ: not ok. Here you write about your work that you have done and why is it important. E.g. write about 
%(a) warehouse automation, (b) why is it important, who can benefit and how (b)what are the components required %to achieve it - (b.1)perception of environment, (b.2)collision detection and avoidance, (b.3)object localization, pose estimation, (b.4)robotics - manipulator. Further write about (c) integration that all the functionalities needs to be integrated both at hardware, software and system level all.
% øø Done?

\section{Motivation}
Logistics is a major building block of the modern society, and increasing labour cost, higher expectations from customers and growth in e-commerce places increasing demands on the logistical system. One part of this system is the many warehouses around the world. These increasing demands have prompted warehouses to implement new automation technology in order to stay afloat \cite{CaiKai2020Wabl}. Warehouse automation is a broad term that involves a lot of different systems and solutions. In the context of this thesis, warehouse automation will refer to object fetching using autonomously navigating mobile manipulator robots. The motivation for this thesis is to investigate how a mobile robot could be set up so that it is capable of preforming warehouse automation tasks. The project requires several design considerations both mechanically and in terms of software, which would be an enticing challenge for a Mechatronics student. Additionally, the resulting system could be a great educational platform for future students to expand and improve upon the presented solution, but also for students to develop new solutions.

The idea is to have a mobile robot capable of navigating around in a traditional warehouse and fetching objects to be delivered at a predefined "delivery station". For a mobile robot to achieve this level of autonomy, several key hardware and software components are needed. The task can be divided into three categories: autonomous navigation, pick and place and a top-level system that interfaces with the two others to orchestrate the mobile robots behaviour. Autonomous Navigation can be divided into 4 subcategories and Pick and Place can be divided into two subcategories. These are further explained below.

\subsection{Autonomous Navigation}\label{sec:I:AutonomousNavigation}
Autonomous navigation in this context refers to autonomous mobile robot navigation. According to \cite{SiegwartRoland2011Itam}, navigation is one most challenging tasks of a mobile robot. Before autonomous navigation can be achieved, there are four capabilities that needs to be set up \cite{SiegwartRoland2011Itam}: Motion control, the robot needs to be able to move itself in a predictable fashion. Perception, the robot needs to be able to perceive its surroundings on a format that is useful for the robot. Localisation, the robot has to determine its localisation in its environment. Cognition, the robot needs to be able to determine how to act and react to its environment in order to achieve its goal, also known as navigation.

Motion Control refers to moving a mobile robot in a predictable manner based on control inputs. This starts with the mechanical design of the mobile robot. A mobile robot for warehouse automation should be designed with this in mind. Additionally it needs to be adequately sized to accommodate all the components of the system.  It also needs to have a large enough payload capacity to carry the objects it fetches. Finally, it needs a robust and accurate control system.

In order to achieve autonomous navigation, a mobile robot needs some kind of perception system which gives it information about the surrounding area. This system will become the eyes of the robot, giving it the information it needs to localise itself and react to a dynamically changing environment.

With a perception and a motion control system, the robot is given the tools to be able to localise itself and map it's environment. The process of doing this is called Simultaneous Localisation And Mapping(SLAM). This is an advanced problem in reality as the information provided from the sensor data could have inherent inaccuracies.  For warehouse automation, localisation and mapping should be robust and accurate enough so that the mobile robot can successfully use this information for navigation.

With a working localisation and mapping system, the mobile robot needs to be able to plan it's route and react to a dynamically changing environment. If evasive actions are taken, or if the environment has changed, the mobile robot should be able to re-plan it's route accordingly. This is referred to as the cognitive part of the system, or the navigation system itself.

\subsection{Pick and Place} \label{sec:I:PAP}
With a mobile robot successfully navigating autonomously around the warehouse, it now needs means of picking the object it is tasked to fetch. However, the robot needs to know the position of the object for the picking operation to be successful. This can be done by having the objects at fixed locations, but this is not a very flexible solution and it puts tremendous precision requirements on the navigation system. A solution that has the potential to be more flexible and robust, is by the use of a machine vision system that can detect and estimate the pose of the object. computer vision solutions for object detection and pose estimation are extensively studied and there are numerous solutions to this problem.

Assuming that the pose of the object is known to the pick and place system, the robot now needs a robotic arm, often called a manipulator, to be able to pick the object. Robotic manipulators have been around for some decades and there are countless different shapes and sizes to choose from. In the case of warehouse automation, the manipulator should be small enough to be mounted on a mobile robot, yet strong enough to lift the objects it is tasked to manipulate. The manipulator should be equipped with some kind of gripping mechanism that is suited for grasping the types of objects it is presented with.

%With the 6-Degrees Of Freedom(6-DOF) robotic arm probably being the most popular choice for most pick and place tasks.

\section{Objective} \label{sec:I:Objective}
An autonomous navigating mobile robot has been set up previously at UiA, using a Unmanned Ground Vehicle (UGV) as a platform \cite{MAS513Rep}. The mobile robot also had a simple implementation of a lane detection and following algorithm intended for roads, based on a pre-trained Convolutional Neural Network (CNN) trained for road lane detection. It should also be noted that there are two other students, Didrik Robsrud and Lars Muggerud, working on their own theses on the same robotic platform \cite{robsrud2023}, \cite{muggerud2023}. 

Using the mobile robot from the MAS513-G project as a starting point, the objective of this thesis is to modify this setup so that it is able to preform warehouse automation tasks. The solution should be modular so that it is possible for future students to pick up on the work. This involves modifying the autonomous navigation system to accommodate the addition of a manipulator, and developing a pick and place system and a top-level system.

\subsection{Warehouse Automation Benchmark Scenario} \label{sec:I:O:WarehouseScenario}
In order to better define the capabilities that this project aims to achieve, a warehouse automation scenario is defined. This scenario describes a typical warehouse automation task that involves both autonomous navigation and pick and place using a vision guided robotic system. The following list will provide a better understanding of the test scenario and the order in which its different parts should be performed.
\begin{enumerate}
    \item Autonomously navigate from any position within the workspace of the mobile robot to an arbitrarily chosen pick position (also within the workspace of the mobile robot). Given that a route is viable.
    \item Preform object detection and pose estimation of a chosen pick object using a computer vision system
    \item Pick the defined object using a manipulator mounted on the mobile robot.
    \item Autonomously navigate to a predefined place position
    \item Drop of the picked object at the place position
    \item Autonomously navigate to a predefined home position
\end{enumerate}




% Although most of the work in this thesis regards constructing a Pick and Place system and a Top Level system for the existing mobile robotic platforms, the Autonomous Navigation system is also altered in order to accommodate for new systems and to improve navigational performance. 

