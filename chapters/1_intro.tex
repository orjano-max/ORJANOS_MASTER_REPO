\chapter{Introduction}

Autonomous robotics has been gaining massive traction over the last decade. This increased interest in autonomous robotics has sparked inventions like Robot Operating System(ROS). ROS, and its successor Robot Operating System 2(ROS 2), is an open source operating system made give developers the tools necessary to implement advanced robotic applications without the need for comprehensive knowledge about the inner workings of for example 3D Light Detection and Ranging(LiDAR) sensors or Permanent Magnet Synchronous Motors(PMSM).  This opens the door to develop autonomous warehouse automation systems where autonomous Unmanned Ground Vehicles(UGV), could navigate a standard warehouse and fetch items using a mounted robotic manipulator. This thesis presents a proof of concept for using differential drive robots together with machine vision and robotic manipulators to preform autonomous warehouse automation tasks.


% AJ not ok. Here you write about your work that you have done and why is it important. E.g. write about 

(a) warehouse aotomation, (b)what are the components required to achieve it - 


\section{Autonomous Navigation}\label{sec:I:AutonomousNavigation}
Autonomous navigation in this context refers to autonomous mobile robot navigation. According to \cite{SiegwartRoland2011Itam}, navigation is one most challenging tasks of a mobile robot. Before autonomous navigation can be achieved, there are four "building blocks of navigation" that has to be achieved \cite{SiegwartRoland2011Itam}. Perception, the robot needs to be able to perceive its surroundings on a format that is useful for the robot. Localisation, the robot has to determine its localisation in its environment. Cognition, the robot needs to be able to determine how to act and react to its environment in order to achieve its goal. Motion control, the robot needs to be able to move itself in a predictable fashion.