\chapter*{Abstract}


The use of mobile robot-based warehouse automation systems is increasing with the increasing logistic demands of modern society. Manufacturers now provide out-of-the box solutions for mobile manipulator robots for use in warehouse automation and research and development. These robotic systems are made up of a mobile robot with a mounted manipulator and provides a platform for autonomous warehouse automation.

This thesis aims to retrofit an exiting mobile robotic system with a vision guided manipulator to give it the same mobile manipulator abilities as these mobile robotic systems, and in addition, fit it with a computer vision system for object detection and pose estimation. The solution is a system that is  divided into three separate parts. The first part is autonomous navigation, which includes the existing autonomous mobile robot and its configuration. Modifications is done to this system in order to fit the work of this thesis. The changes involve mechanical modifications and additional IMU information to localisation. The second part is pick and place, which includes a robotic manipulator and a computer vision system for object detection and pose estimation. The final part is a top-level system that interfaces with the two other parts in order to achieve end-to-end a warehouse automation task.

The proposed method and associated functionalities is tested by creating a warehouse automation scenario at laboratory conditions. The mobile robot successfully autonomously navigate from a start position to an arbitrary pick location within the SLAM-generated map in which the mobile robot locates itself. Then, it uses its manipulator and computer vision to detect a chosen object, estimate its pose and pick this object. Next, the robot autonomously navigates to a pre-defined place location within the SLAM-generated map and drop off the picked object. Finally, the robot autonomously navigate back to its pre-defined home position. The test scenario is run on three different experimental setups, a physical test setup of the system, a simulated test setup of the system, and a simulation of a TurtleBot3.

The experiments shows that the proposed solution is able to perform a similar task as the one given by the test scenario. Links to videos of the three experiments are provided in appendix \ref{A:ExperimentVideos} to better demonstrate the performance of the solution.

%aj nice. well done.