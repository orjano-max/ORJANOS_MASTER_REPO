\chapter*{Sammendrag}
Med de økende logistiske kravene i dagens moderne samfunn, har også bruken av mobile roboter innen varehusautomasjon økt. Nå til dags, leverer robotikkprodusenter ferdige løsninger med mobile manipulator roboter til bruk innen for eksempel varehusautomasjon, og forskning- og utvikling. Disse systemene består av en mobil robot med en montert robotmanipulator, noe som gir em plattform der man kan utvikle et varehusautomasjonssystem. 

Målet ved denne oppgaven er å modifisere en eksisterende mobil robot for å oppnå de samme egenskapene som ferdige mobile manipulator systemer. I tillegg skal den utstyres med et maskinsynssystem for objekt deteksjon og posisjon- og orienterings-estimering. Den foreslåtte løsningen er et system som kan deles opp i tre separate deler. Den første delen er autonom navigering, som består av den eksisterende autonome mobile robotsystemet og konfigurasjonen av dette. Dette systemet har blitt tilpasset slik at det passer bedre sammen med resten av systemene i denne oppgaven. Den andre delen av den foreslåtte løsningen er plukking og plasserings-systemet, som består av en robotmanipulator og et maskinsynssystem for objekt-gjenkjenning og posisjon- og orienterings-estimering. Den siste delen av den foreslåtte løsningen er et toppnivå system som kommuniserer med de andre delene for å orkestrere en varehusautomasjons-oppgave.


\textbf{Se over, under her}
Ett senario som beskriver en varehusautomasjons-oppgave, har blitt definert for å teste funksjonaliteten til den foreslåtte løsningen. Den mobile manipulator-roboten skal navigere autonomt fra ett startpunkt til en vilkårlig valgt posisjon i robotens arbeidsområde. Deretter skal roboten bruke manipulatoren og maskinsynsystemet til å gjenkjenne et valgt objekt, estimere objektets posisjon- og orientering og plukke det opp. Neste oppgave er å autonomt navigere til en forhåndsdefinert plasseringsplass og legge ned det plukkede objektet. Til slutt skal roboten autonomt navigere til sin forhåndsdefinerte hjemme posisjon. Testsenarioet utføres på tre forskjellige eksperimentelle oppsett; et fysisk testoppsett, et simulert testoppsett, og et simulert testoppsett av en TurtleBot3.

Eksperimentene viser at den foreslåtte løsningen klarer å gjennomføre en lignende oppgave som det definerte scenarioet. 
Videoer av de tre eksperimentene er gitt for å bedre demonstrere ytelsen til løsningen: \href{https://youtu.be/hxrZh7bj16A}{Fysisk Eksperiment}, \href{https://youtu.be/po9pRVNtn78}{Simulert Eksperiment}, \href{https://youtu.be/_Dy3rSTHWYo}{TurtleBot3 Simulert Eksperiment}
